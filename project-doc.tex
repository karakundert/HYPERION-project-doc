\documentclass[11pt]{report}
\newcommand{\thetitle}{HYPERION Project Book}
\newcommand{\theauthor}{Kara Kundert}
\newcommand{\theauthorsemail}{kkundert@berkeley.edu}
\newcommand{\thedate}{Spring 2017}
% the following controls some aspects of how the text is displayed on the page
\setlength{\textwidth}{6.5in}
\setlength{\textheight}{8.25in}
\setlength{\oddsidemargin}{0in}

% set up the page headers and footers
\usepackage{fancyhdr}
    \pagestyle{fancy}
    \lhead{\sffamily\slshape\small\thetitle}
    \rhead{\sffamily\small\theauthor}
    \cfoot{\sffamily\slshape\small\thepage}

% support display of graphics
\usepackage{graphicx}

% the following control some aspects of how paragraphs are displayed
\parindent=0pt
\parskip=2ex

% import library of technical symbols
\usepackage{amsmath,amssymb,latexsym}

% import bibliography tools
\usepackage{natbib}
\citestyle{aa}

\begin{document}
% print the title in san-serif font, in bold, in huge characters
\title{
    \sffamily\bfseries\huge
    \thetitle \\
}
% print the author in san-serif font
\author{
    \sffamily\theauthor \\
    \sffamily\theauthorsemail
}
\date{\thedate}
\maketitle
\sloppy

\tableofcontents{}

\chapter{Introduction}

\section{Purpose}

The HYPERION Project Book aims to provide a complete overview of the HYPERION 
project, including current progress and future goals.

\section{Overview}

The Hydrogen Probe of the Epoch of Reionization (HYPERION) is a specialized 
low-frequency interferometer to study the Epoch of Reionization through the 
spatial monopole of the 21cm brightness temperature of neutral hydrogen as a 
function of redshift (i.e. the ``global signal"). It is being developed at the 
University of California, Berkeley in the Radio Astronomy Laboratory. It is 
being funded through the NSF Faculty Early Career Development (CAREER) Grant 
awarded to Aaron Parsons (Award No. 1352519).

\subsection{Science}

The goal of HYPERION is to detect the monopole reionization signal. A detection 
of this signal could give key insights into the physics and development of our 
early universe, including information on the formation of the first stars and 
galaxies.

Refer to~\citep{pritchard-loeb2010, pritchard-loeb2012, gnedin2004}.  

\subsection{Design}

The Epoch of Reionization is one of the most exciting frontiers of modern day 
cosmology, and one of the most challenging to explore. Over the past ten years, 
astronomers have sought to detect the Epoch of Reionization with many different 
experiments, and primarily focused on using redshifted 21cm emissions to map 
out this period of the universe's history.  In the case of PAPER and HERA, 
astronomers have built specialized low-frequency interferometers in order to 
observe the overall power spectrum of hydrogen in the early universe.  The 
interferometric honeycomb design of HERA also enables the observer to 
potentially image growing ionization bubbles throughout the Epoch of 
Reionization. The LOFAR instrument also seeks to measure this signal through a 
more generalized instrument, with much longer baselines (on the order of 
kilometers compared to PAPER/HERA's baselines, which are on the order of 
decameters). 

Astronomers have also sought to measure the sky-averaged 21cm signal, as a way 
of quantifying the overall neutral hydrogen content in the universe through the 
use of specialized single-dish experiments. The single dish is an important, if 
fraught, design choice as it enables direct sampling of the spatial monopole 
21cm signal (i.e. the all-sky average signal). The trouble with single dish 
experiments is that it is extremely difficult to effectively calibrate every 
source of noise that the instrument itself contributes to the overall system 
noise, particularly in cases where that noise is frequency dependent. One of 
the main benefits of interferometric designs is that the vast majority of 
instrumental noise largely averages itself out, which abates the very strict 
calibration requirements demanded by Epoch of Reionization studies.

The trouble with using an interferometer to observe the 21cm global signal is 
that, speaking from a purely Fourier background, an interferometer cannot be 
used to directly sample the monopole term, as interferometers work by imposing 
spatial frequencies across the sky which are then combined to form images. It 
is clear that a monopole signal, such as the 21cm global signal of 
reionization, would integrate to zero over any set of observed spatial 
frequencies, in the case of a perfect interferometer able to sample every mode 
and see the full 360$^\circ$ sky.  However, HYPERION will circumnavigate this 
issue by using absorber baffles to impose an artificial horizon on the sky.  
This horizon interrupts the flat nature of the spatial monopole, which creates 
leakage from the DC-mode into modes with non-zero interferometric spacings.  As 
discussed in~\cite{presley2015}, the optimal spacing between elements is 
approximately one wavelength. While the baffle structure will likely force a 
larger separation, one consideration to be made during design is ensuring that 
the spacing remains as tightly packed as possible to maximize our sensitivity 
to the monopole term.

Refer to~\citep{presley2015, liu2013, venumadhav2016} for more rigorous 
treatments of the above design parameters.

\chapter{Requirements and Specifications}

These are the requirements and specifications of the HYPERION instrument. 

The antennas will be constructed with a frequency range of 35.36 -- 141.2 MHz 
and a science band of 50 -- 100 MHz. This frequency range optimizes the ability 
to see the change in the 21cm global signal temperature as a function of 
frequency, as can be seen in Fig. 9 of~\citep{pritchard-loeb2010}. 

The brightness temperature of the galaxy is very high in our science band, with 
T$_{sky}$ = 2000 K with an observing frequency of $\nu = 70$ MHz. Factoring in 
the absorber baffle structure and a renormalization to keep T$_{21}$ constant, 
the artificial horizon imposed by the absorber baffles must let in at least 
$20\%$ of the sky in order to maintain a system temperature T$_{sys} < 6000$ 
K~\citep{kundert2016}.  This same test indicated that the amplifier temperature 
is a relatively small contributor to the overall system temperature, and no 
special measures need to be taken to minimize it.

The ADC needs at least -5.5dBm in order to get RMS counts of 12.5 with Gaussian 
noise. Assuming $10\%$ return loss, the antenna will be receiving between 
-87.95 dBm at $\nu = 50$ MHz to -95.48 at $\nu = 100$ MHz. This leads to a 
requirement for at least 18dB of gain~\citep{day2016}.

Refer to~\citep{sys-specs, kundert2016}.

\chapter{Architecture}

This is the overall architecture and work breakdown structure (WBS) of the 
HYPERION instrument.

\section{Element}

\subsection{Overview}

using styrofoam as base for antenna with dielectric constant $\kappa \simeq 1$, 
needs to be painted with latex/water-based paint due to UV instability

\subsection{Requirements}

\subsection{Antenna}

\subsubsection{Overview}

Need an antennas design with a frequency independent broadband beam, frequency 
range of antenna design is 35.36 - 141.42 MHz in order to have flat response 
and minimize edge effects across science band of 50-100 MHz.

Review~\citep{sys-specs}.

\subsubsection{Requirements}

\subsection{Absorber}

Need about 10 dB attenuation for reflective components, need at least 33 dB 
attenuation for diffractive and transmissive components (diffraction over top 
of baffles, transmission between antennas/baffles)

\subsubsection{Overview}

\subsubsection{Requirements}

\subsection{Work}

\subsubsection{Needs Doing}
\begin{itemize}
 \item finalize antenna design and logistics
 \item simulate baffle structures to determine beam shaping properties
 \item finalize baffle design and logistics
 \item determine absorber properties -- ferrite, foam, resistive mesh, 
  charcoal...
 \item take beam measurements
 \item CEM model antenna + ground -- determine best height for antenna to stand
 \item CEM model antenna + absorptive ground -- what effects?
 \item CEM model of absorber baffle structures
 \item characterize fat dipole antenna characteristics -- return loss, beam 
  shape and frequency dependence
\end{itemize}

\subsubsection{Already Completed}

\section{Frontend}

\subsection{Overview}

PCB board + balun with 1:1 ratio, LNA with noise factor 1.1-1.2, voltage 
regulator, plus various caps, resistors, inductors for (DC block?) and 
impedance matching

\subsection{Requirements}

Refer to gain v. loss calculations, HYPERION sys specs doc, and HYPERION Memo 
\#1

\subsection{Receiver}

\subsubsection{Overview}

Two stage filtering (low pass and high pass filter x2, may be amplifier between 
stages) + amplifier after filtering + bias-t (power the pcb board maybe) + any 
needed attenuation + battery for receiver box

\subsubsection{Requirements}

Filter to science band, need enough gain (refer to gain v. loss calculations) 
to get to -5.5dBm at ADC input (including cable loss)~\citep{day2016}.

\subsection{Work}

\subsubsection{Needs Doing}
\begin{itemize}
 \item PCB board design
 \item figure out caps, resistors, inductors needed for matching
 \item power distribution for board
 \item design chassis (weatherproof)
 \item figure out cable strain relief
 \item design shielded chassis
 \item figure out which amplifiers to use
 \item figure out which low pass filter to use
 \item figure if we need bias-t and if so, which to use
 \item figure out power consumption of front end
 \item order batteries
 \item explore node design for receiver cards
 \item look into EM properties of 3d printer filaments for chassis design
 \item figure out array design and placement
 \item hire electrical engineering masters student to design perfect match LNA 
  to replace balun + LNA system?
 \item maybe look into hiring undergrad for LNA design
 \item work with Ali Niknejad to hire BWRC masters student to create custom LNA
\end{itemize}

\subsubsection{Already Completed}
\begin{itemize}
 \item order LNA and balun and LNA-balun boards
 \item figure out which high pass filter to use in receiver box
\end{itemize}

\section{Backend}

\subsection{Overview}

Convert from analog to digital signal, SNAP board correlator, RF shielded box, 
powered by battery

\subsection{Requirements}

RMS ADC counts of about 12

\subsection{Work}

\subsubsection{Needs Doing}
\begin{itemize}
 \item finalize correlator design (check Rachel PoCo documents? Eddie SNAP 
  documents?)
 \item power distribution network design
 \item finish chassis design or order pre-made SNAP chassis (talk to SCI-HI 
  folks)
\end{itemize}

\subsubsection{Already Completed}

\section{Computing}

\subsection{Overview}

\subsection{Requirements}

\subsection{Work}

%\subsubsection{Needs Doing}

\section{Analysis}

\subsection{Overview}

\subsection{Requirements}

\subsection{Work}

\subsubsection{Needs Doing}
\begin{itemize}
 \item do data analysis of PAPER-128 to search for global signal, set up 
  initial tools
\end{itemize}


\bibliography{/home/kara/documents/hyperion/hyperion}{}
\bibliographystyle{apj}

\end{document}

